\documentclass[presentation]{beamer}
\newcommand{\lectnum}{10}
\newcommand{\lectitle}{Java Collections Framework}
\usepackage{../oop-slides}

\begin{document}

\frame[label=coverpage]{\titlepage}

\ackpage{}


\newcommand{\ecl}{\eclipsepath{p10collections}}



\fr{Outline}{
  \bl{Goal della lezione}{\iz{
  \item Illustrare la struttura del Java Collections Framework
  \item Mostrare gli utilizzi delle funzonalità base
  \item Discutere alcune tecniche di programmazione correlate
  }}
  \bl{Argomenti}{\iz{
  \item Presentazione Java Collections Framework
  \item Iteratori e foreach
  \item Collezioni, Liste e Set
  \item \cil{HashSet} e \cil{TreeSet}
  %\item \cil{ArrayList} e \cil{LinkedList}
  %\item Funzioni di utilità in \cil{Arrays} e \cil{Collections}
  }}
}

\section[JCF]{Java Collections Framework}

\fr{Java Collections Framework}{
  \bl{Java Collections Framework (JCF)}{\iz{
    \item \`E una libreria del linguaggio Java
    \item \`E una parte del package \cil{java.util}
    \item Gestisce strutture dati (o collezioni) e relativi algoritmi
  }}
  \bl{Importanza pratica}{\iz{
    \item Virtualmente ogni sistema fa uso di collezioni di oggetti
    \item Conoscerne struttura e dettagli vi farà programmatori migliori
  }}
  \bl{Importanza didattica}{\iz{
    \item Fornisce ottimi esempi di uso di composizione, ereditarietà, genericità
    \item Mette in pratica pattern di programmazione di interesse
    \item Impatta su alcuni aspetti del linguaggio da approfondire
  }}
}

\fr{JCF -- struttura complessiva}{
  \fg{height=0.75\textheight}{img/tax1.png}
}

\fr{JCF -- struttura riorganizzata}{
  \fg{height=0.75\textheight}{img/tax.png}
}


\fr{JCF -- alcuni aspetti generali}{
  \bl{\`E complessivamente piuttosto articolato}{\iz{
    \item Un nostro obbiettivo è quello di isolare una sua sottoparte di interesse
    \item Identificando e motivando le funzionalità prodotte
  }}
  \bl{Due tipi di collection, ognuna con varie incarnazioni}{\iz{
    \item \cil{Collection} -- contenitore di elementi atomici{\iz{
      \item 3 sottotipi: \cil{List} (sequenze), \cil{Set} (no duplicazioni), \cil{Queue}
    }}
    \item \cil{Map} -- contenitore di coppie chiave-valore
  }}
  \bl{Interfacce/classi di interesse:}{\iz{
    \item Interfacce: \cil{Collection}, \cil{List}, \cil{Set}, \cil{Iterator}, \cil{Comparable}
    \item Classi collection: \cil{ArrayList}, \cil{LinkedList}, \cil{HashSet}, \cil{HashMap}
    \item Classi con funzionalità: \cil{Collections}, \cil{Arrays}
  }}  
}

\fr{Una nota su eccezioni e JCF}{
  \bl{Eccezioni: un argomento che tratteremo in dettaglio}{
    Un meccanismo usato per gestire eventi ritenuti fuori dalla normale esecuzione (errori), ossia per dichiararli, lanciarli, intercettarli
  }
  \bl{JCF e eccezioni}{\iz{
    \item Ogni collection ha sue regole di funzionamento, e non ammette certe operazioni che richiedono controlli a tempo di esecuzione (ad esempio, certe collezioni sono immutabili, e non si può tentare di scriverci)
    \item Molti metodi dichiarano che possono lanciare eccezioni -- ma possiamo non preoccuparcene per ora
  }}
}

\section{Iteratori e foreach}

\fr{Foreach}{
  \bl{Costrutto foreach}{\iz{
    \item Abbiamo visto che può essere usato per iterare su un array in modo più astratto (compatto, leggibile){\iz{
      \item \cil{for(int i: array)\{...\}}
    }}
    \item Java fornisce anche un meccanismo per usare il foreach su qualunque collection, in particolare, su qualunque oggetto che implementa l'interfaccia \cil{java.lang.Iterable<X>}
  }}
  \bl{\cil{Iterable} e \cil{Iterator}}{\iz{
    \item L'interfaccia \cil{Iterable} ha un metodo per generare e restituire un (nuovo) \cil{Iterator}
    \item Un iteratore è un oggetto con metodi \cil{next()}, \cil{hasNext()} (e \cil{remove()})
    \item Dato l'oggetto \cil{coll} che implementa \cil{Iterable<T>} allora il foreach diventa:{\iz{
      \item \cil{for(T element: coll)\{...\}}
    }}
  }}
}

\fr{Interfacce per l'iterazione}{
  \sizedcode{\scriptsize}{code/short/Iterable.java}
  \sizedcode{\scriptsize}{code/short/Iterator.java}
  \sizedcode{\scriptsize}{code/short/Collection-Short.java}
}

\fr{Interfacce per l'iterazione -- UML}{
  \fg{height=0.7\textheight}{img/uml-iter.pdf}
}

\fr{Esempio di iterable ad-hoc, e suo uso}{
  \sizedrangedcode{\scriptsize}{3}{100}{\ecl/iterator/Range.java}
  \sizedrangedcode{\scriptsize}{3}{100}{\ecl/iterator/UseRange.java}
}

\fr{Realizzazione del corrispondente iteratore}{
  \sizedrangedcode{\scriptsize}{3}{100}{\ecl/iterator/RangeIterator.java}
}

%\fr{Soluzione alternativa per questo caso specifico}{
%  \sizedcode{\ssmall}{code/iterator/AltRange.java}
%}

\fr{Iteratori e collezioni: preview}{
  \sizedrangedcode{\scriptsize}{3}{100}{\ecl/iterator/UseCollectionIterator.java}
}

\section{Collection, List, Set}

\frs{5}{Interfaccia \Cil{Collection}}{
  \bl{Ruolo di questo tipo di dato}{\iz{
    \item \`E la radice della gerarchia delle collezioni
    \item Rappresenta gruppi di oggetti (duplicati/non, ordinati/non)
    \item Implementata via sottointerfacce (\cil{List} e \cil{Set})
  }}
  \bl{Assunzioni}{\iz{
    \item Definisce operazioni base valide per tutte le collezioni
    \item Assume implicitamente che ogni collezione abbia due costruttori{\iz{
      \item Senza argomenti, che genera una collezione vuota
      \item Che accetta un \cil{Collection}, dal quale prende tutti gli elementi
    }}
    \item Le operazioni di modifica sono tutte ``opzionali''{\iz{
      \item potrebbero lanciare un \cil{UnsupportedOperationException}
    }}
    \item Tutte le operazioni di ricerca lavorano sulla base del metodo \cil{Object.equals()} da chiamare sugli elementi{\iz{
      \item questo metodo accetta un \cil{Object}, influendo su alcuni metodi di \cil{Collection}
    }}
  }}
}


\fr{\Cil{Collection}}{
  \sizedcode{\ssmall}{code/short/Collection.java}
}

\fr{Usare le collezioni}{
  \sizedrangedcode{\scriptsize}{5}{100}{\ecl/collection/UseCollection.java}
}

\fr{Creare collezioni (immutabili) -- Java 9+}{
  \sizedrangedcode{\scriptsize}{5}{100}{\ecl/collection/UseFactories.java}
}

\fr{\Cil{Set} e \Cil{List}}{
  \bl{\cil{Set}}{\iz{
    \item Rappresenta collezioni senza duplicati{\iz{
      \item nessuna coppia di elementi porta \cil{Object.equals()} a dare \cil{true}
      \item non vi sono due elementi \cil{null}
    }}
    \item Non aggiunge metodi rispetto a \cil{Collection}
    \item I metodi di modifica devono rispettare la non duplicazione
  }}
  \bl{\cil{List}}{\iz{
    \item Rappresenta sequenze di elementi
    \item Ha metodi per accedere ad un elemento per posizione (0-based)
    \item Andrebbe scandito via iteratore/foreach, non con indici incrementali
    \item Fornisce un list-iterator che consente varie operazioni aggiuntive
  }}
  \bx{La scelta fra queste due tipologie non dipende da motivi di performance, ma da quale modello di collezione serva!}
}

\fr{\Cil{Set} e \Cil{List}}{
  \sizedcode{\scriptsize}{code/short/List.java}
  \sizedcode{\scriptsize}{code/short/Set.java}
}

\fr{\Cil{ListIterator}}{
  \sizedcode{\scriptsize}{code/short/ListIterator.java}
}

\fr{\Cil{UseListIterator}}{
  \sizedrangedcode{\scriptsize}{4}{100}{\ecl/collection/UseListIterator.java}
}

\fr{Implementazione collezioni -- UML}{
    \fg{height=0.75\textheight}{img/uml-abs.pdf}
}

\fr{Implementazione collezioni: linee guida generali}{
    \bl{Una modalità di progettazione da ricordare}{\iz{
	\item Interfacce: riportano le funzionalità definitorie del concetto
	\item Classi astratte: fattorizzano codice comune alle varie implementazioni
	\item Classi concrete: realizzano le varie implementazioni
    }}
    \bl{Nel codice cliente..}{\iz{
	\item In variabili, argomenti, tipi di ritorno, si usano le interfacce
	\item Le classi concrete solo nella \cil{new}, a parte casi molto particolari
	\item Le classi astratte non si menzionano praticamente mai, solo eventualmente per chi volesse costruire una nuova implementazione
    }}
}


\fr{Implementazione collezioni -- Design space}{
  \bl{Classi astratte}{\iz{
    \item \cil{AbstractCollection}, \cil{AbstractList}, e \cil{AbstractSet}
    \item Realizzano ``scheletri'' di classi per collezioni, corrispondenti alla relative interfacce
    \item Facilitano lo sviluppo di nuove classi aderenti alle interfacce
  }}
  \bl{Un esempio: \cil{AbstractSet}}{\iz{
    \item Per set immutabili, richiede solo di definire \cil{size()} e \cil{iterator()}
    \item Per set mutabili, richiede anche di ridefinire \cil{add()}
    \item Per motivi di performance si potrebbero fare ulteriori override 
  }}
  \bl{Classi concrete.. fra le varie illustreremo:}{\iz{
    \item \cil{HashSet}, \cil{TreeSet}, \cil{ArrayList}, \cil{LinkedList}
    \item La scelta riguarda quasi esculsivamente esigenze di performance
  }}
}

\fr{Esempio di creazione di un nuovo set: \Cil{RangeSet}}{
  \sizedrangedcode{\ssmall}{3}{100}{\ecl/collection/RangeSet.java}
}

\fr{Uso di \Cil{RangeSet}}{
  \sizedrangedcode{\scriptsize}{3}{100}{\ecl/collection/UseRangeSet.java}
}

\section{Implementazioni di \texttt{Set}}

\fr{Implementazioni di \Cil{Set}}{
  \bl{Caratteristiche dei set}{\iz{
    \item Nessun elemento duplicato (nel senso di \cil{Object.equals()})
    \item[$\rightarrow$] Il problema fondamentale è il metodo \cil{contains()}, nelle soluzioni più naive (con iteratore) potrebbe applicare una ricerca sequenziale, e invece si richiedono performance migliori
  }}
  \bl{Approccio 1: \cil{HashSet}}{
   Si usa il metodo \cil{Object.hashCode()} come funzione di \alert{hash}, usata per posizionare gli elementi in uno store di elevate dimensioni
  }
  \bl{Approccio 2: \cil{TreeSet}}{
   Specializzazione di \cil{SortedSet} e di \cil{NavigableSet}. Gli elementi sono ordinati, e quindi organizzabili in un albero (red-black tree) per avere accesso in tempo logaritmico
  }
}

\fr{\Cil{HashSet}}{
  \bl{Idea di base: tecnica di hashing (via \cil{Object.hashCode()})}{\iz{
    \item Si crea un array di elementi più grande del necessario (p.e. almeno il 25\% in più), di dimensione \cil{size}
    \item Aggiunta di un elemento \cil{e}{\iz{
      \item lo si inserisce in posizione \cil{e.hashCode() \% size}
      \item se la posizione è già occupata, lo si inserisce nella prima disponibile
      \item se l'array si riempie va espanso e si deve fare il rehashing
    }}
    \item Ricerca di un elemento \cil{f}{\iz{ 
      \item si guarda a partire da \cil{f.hashCode() \% size}, usando \cil{Object.equals()}
      \item La funzione di hashing deve evitare il più possibile le collisioni
    }}
    \item Risultato: scritture/letture sono $O(1)$ ammortizzato
  }}
  \bl{Dettagli interni}{\iz{
    \item Realizzata tramite \cil{HashMap}, che approfondiremo in futuro
  }}
}

\fr{Costruttori di \Cil{HashSet}}{
  \sizedcode{\scriptsize}{code/short/HashSet.java}
}

\fr{\Cil{equals()} e \Cil{hashCode()}}{
  \bl{La loro corretta implementazione è cruciale}{\iz{
    \item Le classi di libreria di Java sono già OK
    \item \cil{Object} uguaglia lo stesso oggetto e l'hashing restituisce la posizione in memoria..
    \item .. quindi nuove classi devono ridefinire \cil{equals()} e \cil{hashCode()} opportunamente
  }}
  \bl{Quale funzione di hashing?}{\iz{
    \item oggetti \cil{equals} devono avere lo stesso \cil{hashCode}
    \item non è detto il viceversa, ma è opportuno per avere buone performance di \cil{HashSet}
    \item si veda ad esempio: {\tt\small http://en.wikipedia.org/wiki/Java\_hashCode()}
    \item Eclipse fornisce la generazione di un \cil{hashCode} ragionevole (ce ne sono di migliori: \texttt{djb2}, \texttt{murmur3})
  }}
}

\fr{Esempio: \Cil{Persona} pt.1}{
  \sizedrangedcode{\tiny}{3}{34}{\ecl/set/Persona.java}
}

\fr{Esempio: \Cil{Persona} pt.2}{
  \sizedrangedcode{\ssmall}{35}{100}{\ecl/set/Persona.java}
}

\fr{\Cil{UseHashSetPersona}}{
  \sizedrangedcode{\scriptsize}{5}{100}{\ecl/set/UseHashSetPersona.java}
}

\fr{\Cil{TreeSet<E>}}{
  \bl{Specializzazione \cil{NavigableSet} (e \cil{SortedSet})}{\iz{
    \item Assume che esista un ordine fra gli elementi
    \item Quindi ogni elemento ha una sua posizione nell'elenco
    \item Questo consente l'approccio dicotomico alla ricerca
    \item Consente funzioni addizionali, come le iterazioni in un intervallo
  }}
  \bl{Realizzazione ordinamento: due approcci (interno o esterno)}{\en{
    \item O con elementi che implementano direttamente \cil{Comparable}{\iz{
      \item Nota che, p.e., \cil{Integer} implementa \cil{Comparable<Integer>}
    }}
    \item O attraverso un \cil{Comparator} esterno fornito alla \cil{new}
  }}
  \bl{Implementazione \Cil{TreeSet}}{\iz{
    \item Basata su red-black tree (albero binario bilanciato)
    \item Tempo logaritmico per inserimento, cancellazione, e ricerca
  }}
}

\fr{Comparazione ``interna'' agli elementi}{
  \sizedcode{\scriptsize}{code/short/Comparable.java}
  \sizedcode{\scriptsize}{code/Wrappers.java}
  \sizedcode{\scriptsize}{code/SortedPersona.java}
}

\fr{Esempi di comparazione interna}{
  \sizedrangedcode{\ssmall}{3}{100}{\ecl/sortedset/UseComparison.java}
}

\fr{Interfacce \Cil{SortedSet} e \Cil{NavigableSet}}{
  \sizedcode{\scriptsize}{code/short/SortedSet.java}
  \sizedcode{\scriptsize}{code/short/NavigableSet.java}
}


\fr{\Cil{UseTreeSetPersona}: comparazione interna}{
  \sizedrangedcode{\ssmall}{5}{100}{\ecl/sortedset/UseTreeSetPersona.java}
}

\fr{Costruttori di \Cil{TreeSet}, e comparatore ``esterno''}{
  \sizedcode{\scriptsize}{code/short/TreeSet.java}
  \sizedcode{\scriptsize}{code/short/Comparator.java}
}

\fr{Definizione di un comparatore esterno}{
  \sizedrangedcode{\scriptsize}{3}{100}{\ecl/sortedset/PersonaComparator.java}
}

\fr{\Cil{UseTreeSetPersona2}: comparazione esterna}{
  \sizedrangedcode{\ssmall}{3}{100}{\ecl/sortedset/UseTreeSetPersona2.java}
}

\fr{Perché il tipo \Cil{Comparator<? super E>}}{
  \bx{Data una classe \cil{SortedSet<E>} il suo comparatore ha tipo \mbox{\cil{Comparator<? super E>}}, perché non semplicemente \cil{Comparator<E>}?}
  \bl{\`E corretto}{\iz{
    \item \cil{Comparator} ha metodi che hanno \cil{E} solo come argomento
    \item quindi l'uso di \cil{Comparator<? super E>} è una generalizzazione di \cil{Comparator<E>}
  }}
  \bl{\`E utile}{\iz{
    \item Supponiamo di aver costruito un comparatore per \cil{SimpleLamp}, e che questo sia usabile anche per tutte le specializzazioni successivamente costruite (è la situazione tipica)
    \item Anche un \cil{SortedSet<UnlimitedLamp>} deve poter usare il \cil{Comparator<SimpleLamp>}, ma questo è possibile solo grazie al suo tipo atteso \cil{Comparator<? super E>}
  }}
}

\fr{Un esempio di design con le collezioni}{
\bx{Implementare questa interfaccia che modella un archivio persone}
  \sizedrangedcode{\scriptsize}{3}{100}{\ecl/set/Archive.java}
}

\fr{Una soluzione con \Cil{HashSet}}{
  \sizedrangedcode{\ssmall}{5}{100}{\ecl/set/ArchiveImpl.java}
}

\fr{Scenario d'uso dell'archivio}{
  \sizedrangedcode{\scriptsize}{3}{100}{\ecl/set/UseArchive.java}
}

\end{document}
