\documentclass[presentation]{beamer}
\usepackage{../oop-slides}

\title[OOP07: Ereditarietà]{07 \\ Ereditarietà}

\begin{document}

\newcommand{\ecl}{\eclipsepath{p07inheritance}}
%\srcode{\tiny}{3}{32}{\ecl/Point3D.java}

\frame[label=coverpage]{\titlepage}

\fr{Outline}{
  \bl{Goal della lezione}{\iz{
  \item Illustrare il riuso via ereditarietà
  \item Introdurre i vari meccanismi collegati all'ereditarietà
  }}
  \bl{Argomenti}{\iz{
  \item Estensione di classi
  \item Livello d'accesso \cil{protected}
  \item Overriding dei metodi
  \item Gestione dei costruttori e chiamate \cil{super}
  \item Il modificatore \cil{final} su classi e metodi
  }}
}

\section{Riuso via ereditarietà}

\fr{Ereditarietà}{
  \bx{
    \`E un meccanismo che consente di definire una nuova classe \alert{specializzandone} una esistente, ossia ``ereditando'' i suoi campi e metodi (quelli privati non sono visibili), possibilmente/eventualmente modificando/aggiungendo campi/metodi, e quindi riusando codice già scritto e testato.
  }
  \bl{L'ereditarietà è un concetto chiave dell'OO}{\iz{
    \item \`E connesso al meccanismo delle interfacce 
    \item \`E uno degli elementi chiave insieme a incapsulamento e interfacce
    \item Non riguarda solo il riuso di codice, ma influenza anche il polimorfismo conseguente
  }}
  \bl{Solito approccio}{\iz{
    \item Illustreremo i meccanismi base attraverso semplici classi
    \item Successivamente recupereremo l'importanza nei casi reali
    \item Utilizzeremo l'idea di \alert{contatore}
  }}
}

\fr{Scenari di riuso ed estensione}{
  \bx{\iz{
    \item Data una classe, realizzarne un'altra con caratteristiche solo in parte diverse (o nuove)
    \item Come sopra, ma senza disporre dei sorgenti della classe originaria (p.e., la classe di partenza è di libreria)
    \item Data una classe, crearne una più specializzata (ad esempio, più robusta e sicura, anche se più lenta)
    \item Creare gerarchie di classi ossia di comportamenti
  }}
}

\fr{Esempio base: \Cil{Counter}}{
  %\sizedcode{\normalsize}{07/code/pre/Counter.java}
  \srcode{\normalsize}{3}{32}{\ecl/pre/Counter.java}
}

\fr{Uso della classe \Cil{Counter}}{
  \srcode{\normalsize}{3}{32}{\ecl/pre/UseCounter.java}
}

\fr{Una nuova classe: \Cil{MultiCounter}}{
  \srcode{\scriptsize}{3}{32}{\ecl/pre/MultiCounter.java}
}

\fr{Uso della classe \Cil{MultiCounter}}{
  \srcode{\footnotesize}{3}{32}{\ecl/pre/UseMultiCounter.java}
}

\fr{Versione con riuso via composizione: \Cil{MultiCounter2}}{
  \srcode{\scriptsize}{3}{32}{\ecl/pre/MultiCounter2.java}
}

\frs{5}{La necessità di estendere e modificare}{
  \bl{Una tipica situazione}{\iz{
    \item \`E tipico nei progetti software, accorgersi di dover creare anche versioni modificate delle classi esistenti
    \item Appoggiarsi al ``copia e incolla'' di codice come in questo caso è sempre sconsigliabile (principio DRY), perché tende a spargere errori in tutto il codice, e complica la manutenzione
    \item Ottenere riuso via composizione (ossia delegazione) è in generale una \alert{ottima soluzione}.. ma è possibile in alcuni casi fare meglio..
  }}
  \bl{Si usa il meccanismo di ereditarietà}{\iz{
    \item Definizione: \cil{class C extends D \{..\}}
    \item La nuova classe \cil{C} eredita campi/metodi/costruttori non privati di \cil{D}{\iz{
      \item Eredita anche campi/metodi privati, ma non sono accessibili da \cil{C}
      \item I costruttori di \cil{D} non sono direttamente richiamabili con la \cil{new}, bisogna sempre definirne di nuovi
    }}
    \item Terminologia: \cil{D} superclasse, o classe base, o classe padre
    \item Terminologia: \cil{C} sottoclasse, o classe figlio, o specializzazione
    \item Nota: non serve disporre dei sorgenti di \cil{D}, basta il codice binario
  }}
}

\fr{Una nuova versione di \Cil{MultiCounter}}{
  \srcode{\scriptsize}{3}{32}{\ecl/ered/MultiCounter.java}
}

\fr{Razionale}{
  \bl{Ridefiniamo la classe \cil{MultiCounter} come estensione di \cil{Counter}}{\iz{
    \item Definiamo il nuovo metodo \cil{multiIncrement()}
    \item Definiamo il costruttore necessario{\iz{
      \item Il costruttore di una sottoclasse può cominciare con l'istruzione \cil{super}, che chiama un costruttore (non privato) della classe padre
      \item Se non lo fa, si chiama il costruttore di default del padre
      \item Senza costruttori, si ha al solito solo quello di default
    }}
    \item \cil{UseMultiCounter} continua a funzionare!
  }}
  \bl{Il senso della definizione}{\iz{
    \item Un oggetto di \cil{MultiCounter} è simile ad un oggetto di \cil{Counter}{\iz{
      \item Ha i metodi \cil{increment()} e \cil{getValue()}
      \item Ha anche il campo \cil{value} (che in effetti è incrementato), anche se essendo privato è inaccessibile dal codice della classe \cil{MultiCounter}
    }}
    \item Due modifiche necessarie rispetto a \cil{Counter}: metodo \cil{multiIncrement()} e ridefinizione del costruttore
  }}
}

\fr{Notazione UML per l'estensione}{
  \bx{\iz{
    \item Arco a linea continua (punta a triangolo pieno) per la relazione ``\cil{extends}''
    \item Archi raggruppati per migliorare la resa grafica
  }}
  \fg{width=0.45\textwidth}{img/uml-ext.pdf}
}

\fr{Notazione UML -- versione semplificata per il design}{
  \fg{width=0.4\textwidth}{img/uml-ext2.pdf}
}


\fr{Livello d'accesso \Cil{protected}}{
  \bl{Usabile per le proprietà d'una classe}{\iz{
    \item \`E un livello intermedio fra \cil{public} e \cil{private}
    \item Indica che la proprietà (campo, metodo, costruttore) è accessibile dalla classe corrente, da una sottoclasse, e dalle sottoclassi delle sottoclassi (ricorsivamente) -- cavillo: anche da tutto il package
  }}
  \bl{A cosa serve?}{\iz{
    \item Consente alle sottoclassi di accedere ad informazioni della sopraclasse che non si vogliono far vedere agli utilizzatori	
    \item Molto spesso usato a posteriori rimpiazzando un \cil{private}
    \item Molto meglio avere campi privati e getter/setter protetti
  }}
  \bl{Esempio classe \cil{BiCounter} -- contatore bidirezionale}{\iz{
    \item Un contatore con anche il metodo \cil{decrement}
    \item Irrealizzabile senza rendere accessibile il campo \cil{counter}
  }}
}

\fr{Un contatore estendibile: \Cil{ExtendibleCounter}}{
  \srcode{\footnotesize}{3}{32}{\ecl/extendible/ExtendibleCounter.java}
}

\fr{Classe \Cil{MultiCounter}}{
  \srcode{\footnotesize}{3}{32}{\ecl/extendible/MultiCounter.java}
}

\fr{Classe \Cil{BiCounter}}{
  \srcode{\footnotesize}{3}{32}{\ecl/extendible/BiCounter.java}
}

\fr{Overriding di metodi}{
  \bl{Estensione e modifica}{\iz{
    \item Quando si crea una nuova classe per estensione, molto spesso non è sufficiente aggiungere nuove funzionalità
    \item A volte serve anche modificare alcune di quelle disponibili, eventualmente anche stravolgendone il funzionamento originario
    \item Questo è realizzabile riscrivendo nella sottoclasse uno (o più) dei metodi della superclasse (ossia, facendone l'\alert{overriding})
    \item Se necessario, il metodo riscritto può invocare la versione del padre usando il receiver speciale \cil{super}
  }}
  \bl{Esempio}{\iz{
    \item Creare un contatore che, giunto ad un certo limite, non prosegue più
    \item \`E necessario fare overriding del metodo \cil{increment()}
    \item Un ulteriore metodo getter ispeziona il raggiungimento del limite
    
  }} 
}

\fr{Classe \Cil{LimitCounter}}{
  \srcode{\scriptsize}{3}{32}{\ecl/extendible/LimitCounter.java}
}

\fr{Uso della classe \Cil{LimitCounter}}{
  \srcode{\footnotesize}{3}{32}{\ecl/extendible/UseLimitCounter.java}
}


\fr{Notazione UML}{
  \bx{\iz{
    \item I campi/metodi protetti si annotano con un ``\#''
    \item I metodi overridden si riportano anche nella sottoclasse
  }}
  \fg{height=0.6\textheight}{img/uml-ext3.pdf}
}

\section{Uno scenario completo}

\fr{Una applicazione allo scenario domotica}{
  \bl{Elementi}{\iz{
    \item Usiamo \cil{LimitCounter}
    \item Definiamo una \cil{LimitedLamp} (via estensione) che contiene un contatore, e che ha un tempo di vita basato sul numero di accensioni ammesse
    \item Un \cil{EcoDomusController} si compone di $n$ \cil{LimitedLamp}, e ha la possibilità di verificare se tutte le lampadine sono esaurite, e di accendere la lampadina alla quale è rimasto più tempo di vita
  }}
  \bl{Note sulla soluzione}{\iz{
    \item Una alternativa era far sì che \cil{EcoDomusController} componesse $n$ \cil{SimpleLamp} e $n$ \cil{LimitCounter}
    \item \cil{LimitedLamp} realizza alcuni metodi per delegazione al suo contatore
    \item Per ora, non prevediamo l'aspetto di interazione con l'utente
  }}
}

\fr{Diagramma UML complessivo}{
  \fg{height=0.8\textheight}{img/uml-domus.pdf}
}

\fr{\Cil{LimitCounter}}{
  \srcode{\scriptsize}{3}{30}{\ecl/example/LimitCounter.java}
}

\fr{\Cil{SimpleLamp}}{
  \srcode{\scriptsize}{3}{30}{\ecl/example/SimpleLamp.java}
}

\fr{\Cil{LimitedLamp}}{
  \srcode{\ssmall}{3}{30}{\ecl/example/LimitedLamp.java}
}

\fr{\Cil{EcoDomusController} pt 1}{
  \srcode{\ssmall}{3}{29}{\ecl/example/EcoDomusController.java}
}

\fr{\Cil{EcoDomusController} pt 2}{
  \srcode{\ssmall}{30}{55}{\ecl/example/EcoDomusController.java}
}

\fr{\Cil{UseEcoDomusController}}{
  \srcode{\footnotesize}{3}{30}{\ecl/example/UseEcoDomusController.java}
}

\section{Ulteriori dettagli}

\fr{Ereditarietà e costruttori}{
  \bl{Scenario standard}{\iz{
    \item Assumiamo si stia costruendo una catena di sottoclassi
    \item Ogni classe introduce alcuni campi, che si aggiungono a quelli della superclasse a formare la struttura di un oggetto in memoria
  }}
  \bl{Linee guida per la singola classe}{\iz{
    \item Dovrà definire tutti i costruttori necessari, seguendo l'approccio visto
    \item Ogni costruttore dovrà preoccuparsi di:{\en{
      \item Chiamare l'opportuno costruttore padre come prima istruzione (\cil{super}), altrimenti il costruttore di default verrà chiamato, se c'è
      \item Inizializzare propriamente i campi localmente definiti
    }}
  }}
  \bl{Ordine operazioni a seguito di una \cil{new}}{\iz{
    \item Prima si crea l'oggetto con tutti i campi non inizializzati
    \item Il codice dei costruttori sarà eseguito, dalle superclassi in giù
  }}  
}

\fr{Analisi: cosa succede?}{
  \srcode{\scriptsize}{3}{30}{\ecl/analysis/A.java}
  \srcode{\scriptsize}{3}{30}{\ecl/analysis/B.java}
}

\fr{Chiamate di metodo alla superclasse (\Cil{super})}{
  \bl{Chiamate \cil{super}}{\iz{
    \item Una sottoclasse \cil{C} può includere una invocazione del tipo \cil{super.m(..args..)}
    \item Non solo in caso di overriding
    \item Cosa ci aspettiamo succeda?
  }}
  \bl{Semantica}{\iz{
    \item Accade quello che accadrebbe se la classe corrente non avesse il metodo \cil{m}, ossia viene eseguito il metodo \cil{m} della superclasse {\iz{
      \item O, se anche lì assente, quello nella sopraclasse più specifica che lo definisce
    }}
    \item Se tale metodo al suo interno chiama un altro metodo \cil{n} (su \cil{this}), allora si ritorna a considerare la versione più specifica a partire dalla classe di partenza \cil{C}
  }}
}

\fr{Analisi: cosa succede?}{
  \srcode{\scriptsize}{3}{30}{\ecl/analysis/C.java}
  \srcode{\scriptsize}{3}{30}{\ecl/analysis/D.java}
}

\fr{Altra analisi: cosa succede?}{
  \srcode{\scriptsize}{3}{30}{\ecl/analysis/E.java}
  \srcode{\scriptsize}{3}{30}{\ecl/analysis/F.java}
}

\fr{Un esempio: riprendiamo \Cil{LimitCounter}}{
  \srcode{\scriptsize}{3}{30}{\ecl/example/LimitCounter.java}
}

\fr{Un esempio: nuova specializzazione}{
  \bl{Cosa succede chiamando \cil{increment()} su un \cil{UnlimitedCounter}?}{\iz{
    \item Non avendo fatto overriding, si chiama la versione di \cil{LimitCounter}
    \item In \cil{LimitCounter} si chiama \cil{this.isOver()} che chiama \cil{this.getDistanceToLimit()}
    \item La versione di \cil{this.getDistanceToLimit()} eseguita è quella di \cil{UnlimitedCounter}
  }}
  \srcode{\footnotesize}{3}{30}{\ecl/example/UnlimitedCounter.java}
}

\fr{Uso di \Cil{UnlimitedCounter}}{
  \srcode{\scriptsize}{3}{30}{\ecl/example/UseUnlimitedCounter.java}
  %\bx{\iz{
  %  \item \cil{uc.isOver()} chiama la versione del metodo in \cil{LimitCounter}
  %  \item questa chiama il metodo \cil{getRemainingLifeCycle()}, ma di nuovo nella classe \cil{UnlimitedCounter}!
  % }}
}

\fr{La tabella dei metodi virtuali}{
    \bl{Anche detta: vtable, call table, dispatch table}{\iz{
	\item ogni classe \cil{C} ne ha una, ed è accessibile ai suoi oggetti
	\item ad ogni metodo definito (o ereditato) in \cil{C}, associa il codice corrispondente da eseguire, ossia la classe che riporta il body
	\item le chiamate da risolvere con tale tabella sono quelle con \alert{late binding}
	\item è una struttura che rende efficiente il polimorfismo fra classi (che vedremo)
	\item è utile conoscerla anche se non è detto che la JVM usi esattamente tale struttura
	\item fa comprendere il funzionamento di \cil{this.} e \cil{super.}
    }}
    \bl{Esempio}{
	Come sono fatte le tabelle relative alle classi \cil{LimitedCounter} e \cil{UnlimitedCounter} nell'esempio precedente?
    }
}

\fr{Esempio gestione memoria: stack/heap/vtables}{
  \fg{height=0.8\textheight}{img/layout.pdf}
}

\fr{Il modificatore \Cil{final}}{
  \bl{Problema}{\iz{
    \item Tramite l'overriding e le chiamate \cil{super} è possibile prendere classi esistenti e modificarle con grande flessibilità
    \item Questo introduce problemi di sicurezza, specialmente connessi al polimorfismo che vedremo nella prossima lezione
   }}
   \bl{Soluzione: \cil{final}}{\iz{
    \item Oltre che per i campi (e argomenti di funzione o variabili, come già visto), è possibile dichiare \cil{final} anche metodi e intere classi
    \item Un metodo \cil{final} è un metodo che NON può essere ri-definito per overriding
    \item Una classe \cil{final} non può essere estesa
   }}
   \bl{Nelle librerie Java}{\iz{
    \item Moltissime classi sono \cil{final}, ad esempio \cil{String}
   }}
}

\fr{Overriding e controllo d'accesso}{
    \bl{Regole per fare l'overriding di un metodo $M$}{\iz{
	\item La nuova versione deve avere esattamente la stessa signature
	\item È possibile estendere la visibilità di un metodo (da \cil{protected} a \cil{public})
	\item Non è possile limitare la visibilità di un metodo (p.e. da \cil{public} a \cil{protected}, o da \cil{public} a \cil{private})
	\item È possibile indicare il metodo \cil{final}
	\item[$\Rightarrow$] sono tutte conseguenze del principio di sostituibilità
    }}
}


\fr{La classe \Cil{Object}}{
  \bl{Estensione di default}{\iz{
    \item Una classe deve per forza estendere da qualcosa
    \item Se non lo fa, si assume che estenda \cil{java.lang.Object}
    \item Quindi ogni classe eredita (indirettamente) da \cil{Object}
    \item \cil{Object} è la radice della gerarchia di ereditarietà di Java
  }}
  \bl{Classe \cil{Object}}{Fornisce alcuni metodi di utilità generale\iz{
    \item \cil{toString()}, che stampa informazioni sulla classe e la posizione in memoria dell'oggetto
    \item \cil{clone()}, per clonare un oggetto
    \item \cil{equals()} e \cil{hashCode()}, usati nelle collection
    \item \cil{notify()} e \cil{wait()}, usati nella gestione dei thread
    \item ...
    
  }}
}


\end{document}
